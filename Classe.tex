% Exemple de fichier LaTeX pour une présentation de classes
% Projet algorithmique

\documentclass[a4paper,12pt]{article}
\usepackage[utf8]{inputenc}
\usepackage[T1]{fontenc}
\usepackage[french]{babel}
\usepackage{amsmath, amssymb, amsthm}
\usepackage{hyperref}
\usepackage{geometry}
\geometry{margin=2.5cm}

\title{Présentation des Classes du Projet Réseau de Neurones}
\author{Luka, Étienne, Darya, Camille}
\date{\today}

\begin{document}

\maketitle

\section{Classe 1 : \texttt{Neurone}}
\subsection*{Rôle}
Classe permettant de décrire un neurone.

\subsection*{Attributs}
\begin{itemize}
    \item $\omega$ : liste des poids du neurone
    \item $n$ : cardinal de $\omega$
    \item $b$ : biais
    \item $x$ : tableau initialisé à 0 de taille $n$, permettant de stocker l'entrée
    \item $z$ : résultat de $\omega^\top x + b$
    \item $a$ : activation de $z$
\end{itemize}

\subsection*{Méthodes}
\begin{itemize}
    \item \texttt{zupdate()} : met à jour la valeur de $z$ en prenant $x$ comme paramètre
    \item \texttt{bupdate()} : met à jour le biais
    \item \texttt{activation()} : applique une fonction d'activation à $z$
    \item \texttt{forward()} : applique \texttt{zupdate()} puis \texttt{activation()}
\end{itemize}

\section{Classe 2 : \texttt{Layer}}
\subsection*{Rôle}
Classe permettant de décrire une couche de neurones.

\subsection*{Attributs}
\begin{itemize}
    \item \texttt{n\_inputs} : nombre d'entrées
    \item \texttt{n\_neurones} : nombre de neurones
    \item \texttt{tab} : tableau d'objets \texttt{Neurone}
    \item \texttt{f} : tableau des sorties des neurones
\end{itemize}

\subsection*{Méthodes}
\begin{itemize}
    \item \texttt{forward()} : applique la méthode \texttt{forward()} sur chaque neurone de la couche 
\end{itemize}

\section{Classe 3 : \texttt{Reseau\_Neurones}}
\subsection*{Rôle}
Classe permettant de décrire un réseau de neurones.

\subsection*{Attributs}
\begin{itemize}
    \item \texttt{l} : tableau de \texttt{Layer}
    \item \texttt{a} : tableau de tableaux contenant les activations de chaque couche
    \item \texttt{nb\_l} : nombre de couches
\end{itemize}

\subsection*{Méthodes}
\begin{itemize}
    \item \texttt{forward()} : procède à l'activation de l'ensemble des neurones du réseau de neurones
\end{itemize}

\end{document}
