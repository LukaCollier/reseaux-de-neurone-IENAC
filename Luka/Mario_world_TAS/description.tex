\documentclass{article}
\usepackage[utf8]{inputenc}
\usepackage{amsmath}
\usepackage{amssymb}
\usepackage{hyperref}
\usepackage{geometry}
\geometry{margin=2cm}

\title{Résumé du Pont Python--BizHawk et Intégration avec un Réseau de Neurones}
\author{}
\date{}

\begin{document}
\maketitle

\section{Objectif général}
L'objectif est de créer une intelligence artificielle capable de jouer automatiquement à \textbf{Super Mario World} sur un émulateur SNES. Le réseau de neurones est écrit en Python et s'entraîne avec l'optimiseur \textbf{Adam}. Cependant, l'émulateur BizHawk utilise le \textbf{Lua} pour contrôler les entrées. Il faut donc établir un pont entre Python et BizHawk.

\section{Principe du Pont Python--BizHawk}
L'idée générale est de :

\begin{enumerate}
    \item Lancer un script \textbf{Lua dans BizHawk} qui :
    \begin{itemize}
        \item écoute des commandes venant de Python via un socket (TCP par exemple),
        \item lit ces commandes (boutons à presser),
        \item applique les commandes via \texttt{joypad.set},
        \item avance l'émulation image par image via \texttt{emu.frameadvance()}.
    \end{itemize}
    \item Depuis Python :
    \begin{itemize}
        \item générer des actions via le réseau de neurones,
        \item envoyer les actions à BizHawk sous forme de données (JSON ou tableau),
        \item éventuellement recevoir des informations (RAM, image, reward). 
    \end{itemize}
\end{enumerate}

\section{Principe du contrôle BizHawk sans inclure de code Lua}
Le fichier ne doit pas contenir de script Lua. Cette section décrit donc seulement le principe général, sans aucun extrait de code.

BizHawk utilise des scripts Lua pour recevoir et appliquer des commandes de contrôle. Le rôle de Lua est uniquement :

\begin{itemize}
    \item de recevoir des instructions envoyées par Python (état des boutons),
    \item de les convertir en actions dans l'émulateur via \texttt{joypad.set},
    \item puis de faire avancer l'émulation image par image.
\end{itemize}

Dans ce document, aucun code Lua n'est fourni conformément à votre demande.

\section{Structure minimale du client Python}

\begin{verbatim}
import socket, json

def send_input(buttons):
    s = json.dumps(buttons) + "\n"
    sock.sendall(s.encode())

sock = socket.create_connection(("127.0.0.1", 5555))

# Exemple : Mario avance
send_input({"Right":1, "A":0})
\end{verbatim}

\section{Ce qui est déjà fait en Python}

\begin{itemize}
    \item des couches de neurones (classe \texttt{Layer}),
    \item propagation avant (\texttt{forward}),
    \item rétropropagation manuelle,
    \item optimisation avec Adam entièrement codé à la main,
    \item architecture \texttt{Neural\_Network} supportant différentes activations,
    \item entraînement mini-batch (SGD et Adam).
\end{itemize}

Votre réseau est donc fonctionnel et peut produire des actions à chaque frame.

\section{À faire}
Pour connecter le réseau de neurones à BizHawk, il reste :

\subsection*{1. Définir l'observation envoyée au réseau}
\begin{itemize}
    \item Capturer l'écran depuis BizHawk (via Lua ou un bridge existant), ou
    \item lire des valeurs RAM pertinentes (position de Mario, vitesse, etc.).
\end{itemize}

\subsection*{2. Convertir cette observation en entrée compatible avec votre NN}
\begin{itemize}
    \item normalisation des pixels,
    \item ou simple vecteur d'état si RAM.
\end{itemize}

\subsection*{3. Transformer la sortie du NN en actions bouton}
\begin{itemize}
    \item typiquement : sortie \texttt{softmax} ou seuils pour \texttt{A}, \texttt{B}, \texttt{Left}, etc.,
    \item puis construire un dictionnaire JSON pour Lua.
\end{itemize}

\subsection*{4. Implémenter la boucle Principal RL}
\begin{itemize}
    \item envoyer observation dans \texttt{forward()},
    \item envoyer l'action au script Lua,
    \item récupérer reward + nouvel état,
    \item faire une mise à jour ( Adam).
\end{itemize}

\subsection*{5. Définir une fonction de reward}
\begin{itemize}
    \item progression horizontale de Mario,
    \item éviter les morts,
    \item éventuellement : vitesse, pièces, niveau atteint.
\end{itemize}

\section{Conclusion}
Il ne manque que le \textbf{pont avec BizHawk} et la \textbf{boucle d'entraînement RL} qui utilise ce pont.


\end{itemize}

\end{document}
